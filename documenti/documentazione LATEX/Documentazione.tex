\documentclass{article}

\usepackage{graphicx} % Per immagini
\usepackage{tabularx} % Per creare tabelle styfel
%\setlength\parindent{0pt} % Rimuove indendazioni
\usepackage[utf8x]{inputenc} %serve per i simboli utf8x
\usepackage[italian]{babel}

%\usepackage{subcaption} %serve per mettere le subfigure ESCLUDE SUBFIG

\usepackage{times} % Uncomment to use the Times New Roman font
\usepackage{siunitx}
\usepackage{textgreek}
\usepackage{multirow}
\usepackage{siunitx}
%\usepackage{textcomp}
%\usepackage{fancyhdr}
\usepackage{float} %serve per mettere le immagini in posizione Here
%\usepackage{subfig} %serve per mettere più figure in colonna o in riga
\usepackage[a4paper,total={170mm,257mm},left=25mm,right=25mm,bottom=35mm]{geometry} % per il layout



\begin{document}

\title{DOCUMENTAZIONE PROGETTO INGEGNERIA DEL SOFTWARE}

\author{Cracco Andrea\\Litterini Simone\\Alessio Meneghini }

\date{\today}

\maketitle

\vspace{1.5cm}

\newpage


\section{Descrizione del progetto}

Questo progetto implementa una interfaccia per la gestione dei ricoveri di un reparto di rianimazione.
\`E possibile verificare in tempo reale i dati di monitoraggio delle macchine di supporto vitale (bpm, sbp, dbp, temperatura),
visualizzare gli allarmi e scrivere report, aggiungere e rimuovere ricoveri, visualizzare lo storico dei ricoveri,
prescrivere farmaci e somministrare farmaci.
\`E inoltre possibile osservare i grafici dell'andamento dei parametri nelle ultime 2 ore.\\

Il software \`e stato implementato utilizzando un'architettura client-server, dove sia le interfacce utente che le macchine
di monitoraggio si connettono ad un unico server centrale.\\

L'aggiornamento dei client da parte del server \`e effettuato tramite un sistema ad eventi, in cui ogni client rimane in attesa
di un messaggio dal server in un thread separato, ed appena lo riceve lo processa e richiede un altro messaggio.
In questo modo \`e possibile eseguire l'update in tempo reale sul proprio client delle modifiche effettuate in altri client.\\

L'interfaccia di comunicazione client-server \`e implementata utilizzando le Java RMI, ovvero un framework in grado di eseguire
in modo trasparente metodi di interfacce chiamate dal client ma implementate sul server.\\

L'interfaccia iniziale di comunicazione \`e una factory con cui il server genera le altre sottointerfacce specifiche per ogni finestra.\\

Ogni finestra ha una sua specifica interfaccia con il server, che permette di effettuare soltanto alcune azioni, a seconda della tipologia.

Per il salvataggio dei dati \`e stato utilizzato il database PostgreSQL, creando delle interfacce wrapper di ogni tabella, cos\`i da separare
l'effettiva implementazione e connessione SQL dal resto del server.

Il frontend \`e implementato utilizzando Java Swing, ed ogni finestra \`e creata per svolgere uno scopo (comunicando con il server
tramite un'interfaccia specifica) e termina automaticamente al completamento (es. login, aggiunta ricovero, inserimento prescrizione, ecc...).\\

Gli utenti del software sono gestiti dal primario o da medici che il primario pu\`o nominare amministratori.\\

Per l'aggiunta dei farmaci \`e utilizzato il database online all'indirizzo:\\
https://farmaci.agenziafarmaco.gov.it/bancadatifarmaci/home


\section{USE CASE del software}

	\vspace{0.5cm}

	\begin{figure}[H]

		\includegraphics[width=0.9\textwidth]{documenti/useCase_infermiere.pdf}
		\caption{Use case del medico e dell'infermiere con le varie attività che possono svolgere.}
		\label{usecase_medico_infermiere}

	\end{figure}

	\vspace{0.5cm}

	\begin{figure}[H]

		\includegraphics[width=0.9\textwidth]{documenti/useCase_medici.pdf}
		\caption{Use case delle relative sottoclassi di medico, come per esempio ammnistratore o primario.  E le attività che possono svolgere}
		\label{usecase_medici}

	\end{figure}

	\vspace{0.5cm}


\newpage




\section{USE CASE principali con i relativi diagrammi di attività e di sequenza}

\subsection{"Log-in" use case}

Use case per il Log-in. Descrive la procedura da seguire per eseguire l'autenticazioen di un utente per poter poi entrare nella schermata di monitoraggio vera e propria, differente per tipologia di utente. 

\vspace{1cm}

\begin{center}

	\begin{tabular}{r@{\vspace{0.4cm}}ll}
	

	\hline
	
	\textbf{Template UseCase} & \textbf{          "Log-in" use case } \\

	\hline
	
	\multicolumn{1}{c}{Attori}  & Medico e Infermiere \\

	\cline{1-1}
\hline

	\multicolumn{1}{c}{Pre-Condizioni}  & Il server deve essere attivo per potersi collegare alle macchine ed al database.\\ \cline{1-1}&

La connessione deve essere attiva per la trasmissione dei pacchetti di rete con il server, \\&
Inoltre il server e il Database devono essere online ( presume ci sia la connessione sulla macchina ).\\

\hline
	\multicolumn{1}{c}{Sequenza}    \\ \cline{1-1}&
 1. L’utente inserisce l’username \\&
    2. L’utente inserisce la password \\&
    3. preme il pulsante di “Log-in”\\&
    4. Vengono inviate le credenziali al server e si attende  \\&
	    5. Il server crea fa una domanda al database tramite query SQL \\&
    6. Se nel database c’è un istanza in utente che cincide con username e password, l’autenticazione è garantita \\

\hline
	
\multicolumn{1}{c}{Post-Condizioni}  & Il Frame di autenticazione scompare e viene mostrato il monitor con I pazienti \\ \cline{1-1}
\hline

\multicolumn{1}{c}{Sequenza alternativa}  \\     \cline{1-1}&

1. L’utente non si ricorda più la password \\&
     2. Preme il pulsante di “password dimenticata”\\&
        3. L’utente inserisce una mail valida(collegata al proprio account)  \\&
	    4. L’utente conferma tramite il pulsante  \\&
        5. Il server riceve i dati inseriti e controlla la validità della mail \\&
    6. Se la mail è valida, viene inviata una mail con la nuova password all’utente.\\

\hline

	\end{tabular}

\end{center}


\begin{figure}[H]

		\includegraphics[width=1.1\textwidth]{documenti/Activity_Sequence_diagram_Login.pdf}
		\caption{Activity e sequence diagram del Log-in usecase}
		\label{ActivitySequencediagramLogin}

\end{figure}


\vspace{2cm}




\subsection{"Aggiunta ricovero" use case}

Use case della procedura di aggiunta ricovero, descrive il metodo per aggiungere un nuovo ricovero nel database. Può essere eseguito da infermieri e medici.

\vspace{1cm}

\begin{center}

	\begin{tabular}{r@{\vspace{0.4cm}}ll}
	

	\hline
	
	\textbf{Template UseCase  } &  \textbf{       "Aggiunta ricovero" use case } \\

	\hline
	
	\multicolumn{1}{c}{Attori}  & Infermiere o Medico(in caso non ci siano infermieri) \\

	\cline{1-1}
\hline

	\multicolumn{1}{c}{Pre-Condizioni}  & L’utente deve aver fatto l’autenticazione e deve essere\\ \cline{1-1}&

un utente di tipo infermiere o medico, Inoltre il server e il Database devono essere online\\&
( presume ci sia la connessione sulla macchina ).\\

\hline
	\multicolumn{1}{c}{Sequenza}    \\ \cline{1-1}&
    1. L’utente preme il pulsante verde sul monitor \\&
    2. L’utente deve inserire tutti I dati pertinenti all’aggunta di un nuovo ricovero \\&
    3. Preme il pulsante di aggiunta nuovo ricovero\\&
    4. Il server riceve I dati pertinenti al nuovo paziente da ricoverare  \\&
    5.Il server tramite una query al database in SQL aggiunge se non esiste già il nuovo ricovero  \\

\hline
	
\multicolumn{1}{c}{Post-Condizioni}  &Al monitor dei pazienti viene aggiunto un nuovo PatientMonitor \\ \cline{1-1}&

che mostra Il paziente con I relativi dati e parametri vitali\\
\hline

\multicolumn{1}{c}{Sequenza alternativa}  \\     


\hline

	\end{tabular}

\end{center}

\vspace{2cm}


\begin{figure}[H]

		\includegraphics[width=1.1\textwidth]{documenti/Activity_Sequence_diagram_Ricovero.pdf}
		\caption{Activity e sequence diagram del Log-in usecase}
		\label{Activity_Sequence_diagram_Login}

	\end{figure}


\vspace{2cm}



\subsection{"Gestione utenti" use case}

Use case per la gestione degli utenti, permette ad un amministratore di aggiungere, rimuovere o modificare un utente medico o infermiere nel database degli utenti.

\vspace{1cm}

\begin{center}

	\begin{tabular}{r@{\vspace{0.4cm}}ll}
	

	\hline
	
	\textbf{Template UseCase} & \textbf{          "Gestione utenti" use case } \\

	\hline
	
	\multicolumn{1}{c}{Attori}  & Medico amministratore o primario
 \\

	\cline{1-1}
\hline

	\multicolumn{1}{c}{Pre-Condizioni}  & Essere loggato come amministratore o primario.\\&
il server e il Database devono essere online ( presume ci sia la connessione sulla macchina ).\\

\hline
	\multicolumn{1}{c}{Sequenza}    \\ \cline{1-1}&
 1. Il medico primario o amministratore inserisce i dati dell’utente da aggiungere \\&
    2.    Il medico primario o amministratore preme il pulsante aggiungi; \\&
    3. Il server riceve i dati del nuovo utente\\&
    4. Il server, tramite una query SQL, aggiunge l’utente al database se non è già presente. \\
   
\hline
	
\multicolumn{1}{c}{Post-Condizioni}  &Il nuovo utente è stato aggiunto al database utenti\\ \cline{1-1}
\hline

\multicolumn{1}{c}{Sequenza alternativa (rimozione)}  \\     \cline{1-1}&

1.Il medico primario o amministratore seleziona l’utente da rimuovere \\&
         2. Il medico preme il pulsante di conferma;\\&
            3. Il server con una query SQL rimuove l’utente selezionato dal database.\\
\hline

\multicolumn{1}{c}{Sequenza alternativa (modifica)}  \\     \cline{1-1}&

1. Il medico seleziona l’utente da modificare \\&
     2. l medico inserisce i dati che vuole modificare al posto di quelli presenti\\&
        3. Il server con una query SQL modifica le informazioni relative all’utente se\\& i nuovi dati sono corretti e compatibili  \\&
	    4. L’utente conferma tramite il pulsante  \\

\hline

	\end{tabular}

\end{center}


\begin{figure}[H]

		\includegraphics[width=1.1\textwidth]{documenti/Activity_diagram_getioneUtenti.pdf}
		\caption{Activity diagram "gestione utenti" use case}
		\label{sequence_diagram_getioneUtenti}

	\end{figure}

\vspace{2cm}

\begin{figure}[H]

		\includegraphics[width=0.9\textwidth]{documenti/sequence_diagram_getioneUtenti.pdf}
		\caption{Sequence diagram "gestione utenti" use case}
		\label{Activity_diagram_getioneUtenti}

	\end{figure}



\vspace{2cm}


\subsection{"Prescription" use case}

Use case per la prescrizione farmaci, consente di aggiungere prescrizioni di farmaci ai pazienti nel database nel caso in cui ci si sia loggati come medici e il sistema verifica i privilegi.

\vspace{1cm}

\begin{center}

	\begin{tabular}{r@{\vspace{0.4cm}}ll}
	

	\hline
	
	\textbf{Template UseCase  } &  \textbf{       "Prescription" use case } \\

	\hline
	
	\multicolumn{1}{c}{Attori}  & Medico \\

	\cline{1-1}
\hline

	\multicolumn{1}{c}{Pre-Condizioni}  & L’utente deve aver fatto l’autenticazione e deve essere\\ \cline{1-1}&

un utente di tipo Medico , inoltre server e Database devono essere online ( presume macchina connessa )\\

\hline
	\multicolumn{1}{c}{Sequenza}    \\ \cline{1-1}&
    1.  Il medico clicca cerca farmaco; \\&
    2. Il database dei farmaci ritorna la lista di farmaci \\&
    3. Il medico aggiunge i dati del paziente e invia\\&
    4. Il server invia con una query SQL la prescrizione  \\&
    5.Il database la registra  \\

\hline
	
\multicolumn{1}{c}{Post-Condizioni}  &La prescrizione è registrata sul database correttamente \\ \cline{1-1}
\hline

\multicolumn{1}{c}{Sequenza alternativa}  \\     

\hline

	\end{tabular}

\end{center}




\begin{figure}[H]

		\includegraphics[width=1.1\textwidth]{documenti/Diagrams_prescrizioni.pdf}
		\caption{Activity e sequence diagram del "Prescription" usecase}
		\label{Diagrams_prescrizioni}

	\end{figure}


\vspace{2cm}


\subsection{"Alarm monitor" use case}

Use case per la gestione dei dati dei monitor e dei relativi allarmi, gestisce l’aggiornamento sincronizzato dei monitor con server e client, attiva l’allarme se i valori sono fuori da quelli accettabili, spegne l’allarme quando i valori tornano nei range di tolleranza.

\vspace{1cm}

\begin{center}

	\begin{tabular}{r@{\vspace{0.4cm}}ll}
	

	\hline
	
	\textbf{Template UseCase  } &  \textbf{       "Alarm monitor" use case } \\

	\hline
	
	\multicolumn{1}{c}{Attori}  & Monitor, Server, Client \\

	\cline{1-1}
\hline

	\multicolumn{1}{c}{Pre-Condizioni}  & Monitor e server accesi e connessi\\ 


\hline
	\multicolumn{1}{c}{Sequenza (aggiornamento) }    \\ \cline{1-1}&
    1.  Il monitor aggiorna i propri dati \\&
    2. Il monitor invia i dati aggiornati al server \\&
    3. Il server invia i dati ricevuti ai client connessi in quel momento\\&
    4.Il client si aggiorna con i dati ricevuti dal server  \\
    

\hline
	
\multicolumn{1}{c}{Post-Condizioni}  &Le informazioni visualizzate sul client sono le stesse fornite dal monitor \\ \cline{1-1}
\hline

\multicolumn{1}{c}{Sequenza alternativa (allarme)}     \\ \cline{1-1}&

 1.  Il monitor aggiorna i propri dati \\&
    2.Il monitor rileva valori anormali e invia l’allarme al server \\&
    3. Il server passa l’allarme e i valori al client\\&
    4. Il client aggiorna i dati e segnala l’allarme con il simbolo di pericolo e un suono  \\

\hline

	\end{tabular}

\end{center}




\begin{figure}[H]

		\includegraphics[width=1.1\textwidth]{documenti/Diagrams_monitorAgg.pdf}
		\caption{Activity e sequence diagram del "Alarm monitor" usecase}
		\label{Diagrams_monitorAgg}

	\end{figure}


\vspace{2cm}


\section{Pattern utilizzati}


sd
sa
dasd














\end{document}
